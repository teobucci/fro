\documentclass[10pt,a4paper,twoside,openright]{book}

% contenuto identico a template_dispense.tex
%%%%%%%%%%%%%%%%%%%%%%%%%%%%%%%%%%%%%%%%%
% Template Dispense
%
% Autore:
% Teo Bucci
%
%%%%%%%%%%%%%%%%%%%%%%%%%%%%%%%%%%%%%%%%%

%----------------------------------------------------------------------------------------
%	FONTS
%----------------------------------------------------------------------------------------

\usepackage[T1]{fontenc} % Use 8-bit encoding that has 256 glyphs
\usepackage[utf8]{inputenc} % Required for including letters with accents
\usepackage{dsfont} % per funzione indicatrice

%----------------------------------------------------------------------------------------
%	VARIOUS REQUIRED PACKAGES AND CONFIGURATIONS
%----------------------------------------------------------------------------------------

\usepackage[english]{babel} % Italian language/hyphenation

\usepackage{latexsym}

%\usepackage{amsmath,amsfonts,amssymb,amsthm} % For math equations, theorems, symbols, etc

%----------------------------------------------------------------------------------------
%	FIGURE MATHCHA
%----------------------------------------------------------------------------------------

\usepackage{amsmath}
%\usepackage[fleqn]{amsmath}  % per avere l'allineamento a sinistra delle equazioni
\usepackage{amssymb} % carica anche \usepackage{amsfonts}
\usepackage{amsthm}
\usepackage{tikz}
\usepackage{mathdots}
\usepackage{cancel}
\usepackage{color}
\usepackage{siunitx}
\usepackage{array}
\usepackage{multirow}
%\usepackage{gensymb} % The gensymb package provides a number of ‘generic’ macros, which produce the same output in text and math mode: \degree \celsius \perthousand \ohm \micro
\usepackage{makecell}
\usepackage{tabularx}
\usepackage{booktabs}
\usepackage{caption}\captionsetup{belowskip=12pt,aboveskip=4pt}
\usepackage{subcaption}
\usetikzlibrary{fadings}
\usetikzlibrary{patterns}
\usetikzlibrary{shadows.blur}
\usepackage{placeins} % The placeins package gives the command \FloatBarrier, which will make sure any floats will be put in before this point
\usepackage{flafter}  % The flafter package ensures that floats don't appear until after they appear in the code.

%----------------------------------------------------------------------------------------
%	INCLUSIONE FIGURE
%----------------------------------------------------------------------------------------

\usepackage{import}
\usepackage{pdfpages}
\usepackage{transparent}
\usepackage{xcolor}
\usepackage{graphicx}
\graphicspath{ {./images/} } % Path relative to the main .tex file 
\usepackage{float}

\newcommand{\fg}[3][\relax]{%
  \begin{figure}[H]%[htp]%
    \centering
    \captionsetup{width=0.7\textwidth}
      \includegraphics[width = #2\textwidth]{#3}%
      \ifx\relax#1\else\caption{#1}\fi
      \label{#3}
  \end{figure}%
  \FloatBarrier%
}

%----------------------------------------------------------------------------------------
%	PARAGRAFI, INTERLINEA E MARGINE
%----------------------------------------------------------------------------------------

\usepackage[none]{hyphenat} % Per non far andare a capo le parole con il trattino

\emergencystretch 3em % Per evitare che il testo vada oltre i margini

% \parindent 0ex % TOGLIE INTENDAMENTO PARAGRAFI, E' INCLUSO NEL PACCHETTO \parskip
% \setlength{\parindent}{4em} % VARIANTE DI QUELLO SOPRA

% \setlength{\parskip}{\baselineskip} % CAMBIA SPAZIO TRA PARAGRAFI (POSSO METTERE ANCHE 1em) INCLUSA LA TABLE OF CONTENTS, PERTANTO USO IL COMANDO CHE SEGUE

\usepackage[skip=0.2\baselineskip+2pt]{parskip}

% \renewcommand{\baselinestretch}{1.5} % CAMBIA INTERLINEA

%----------------------------------------------------------------------------------------
%	HEADERS AND FOOTERS
%----------------------------------------------------------------------------------------

\usepackage{fancyhdr}

\pagestyle{empty} % Il fancy serve a partire al primo capitolo
\fancyhead{} % Pulisci header
\fancyfoot{} % Pulisci footer

\fancyhead[RE]{\nouppercase{\leftmark}}
%\fancyhead[LO]{\emph{Trascrizioni di Teo Bucci}}
\fancyhead[LO]{\nouppercase{\rightmark}}
\fancyhead[LE,RO]{\thepage}

% Removes the header from odd empty pages at the end of chapters
\makeatletter
\renewcommand{\cleardoublepage}{
\clearpage\ifodd\c@page\else
\hbox{}
\vspace*{\fill}
\thispagestyle{empty}
\newpage
\fi}

%----------------------------------------------------------------------------------------
%	COMANDI PERSONALIZZATI
%----------------------------------------------------------------------------------------

\newcommand{\Tau}{\mathcal{T}}
\newcommand{\sgn}{\mathop{\mathrm{sgn}}} % segno di funzione
%\renewcommand{\epsilon}{\varepsilon}
%\renewcommand{\theta}{\vartheta}
%\renewcommand{\rho}{\varrho}
%\renewcommand{\phi}{\varphi}
\newcommand{\degree}{^\circ\text{C}} % SIMBOLO GRADI
\newcommand{\notimplies}{\mathrel{{\ooalign{\hidewidth$\not\phantom{=}$\hidewidth\cr$\implies$}}}}
\renewcommand{\qed}{\tag*{$\blacksquare$}}
\newcommand{\latex}{\LaTeX\xspace}
\newcommand{\tex}{\TeX\xspace}
\newcommand{\questeq}{\overset{?}{=}} % è vero che?
\renewcommand{\leqslant}{\leq}
\renewcommand{\geqslant}{\geq}
\newcommand{\complementary}{^{\mathrm{C}}}
\renewcommand{\emptyset}{\varnothing} % simbolo insieme vuoto
\newcommand{\Bot}{\perp \!\!\! \perp} % indipendenza
\newcommand{\boxedText}[1]{\noindent\fbox{\parbox{\textwidth}{#1}}}

\usepackage{mathtools} % Serve per i due comandi dopo
\DeclarePairedDelimiter{\abs}{\lvert}{\rvert} % CREA UN COMANDO abs()
\DeclarePairedDelimiter{\norma}{\lVert}{\rVert} % CREA UN COMANDO norma()

% Simbolo di integrale barrato

\makeatletter
\newcommand*{\mint}[1]{%
  % #1: overlay symbol
  \mint@l{#1}{}%
}
\newcommand*{\mint@l}[2]{%
  % #1: overlay symbol
  % #2: limits
  \@ifnextchar\limits{%
    \mint@l{#1}%
  }{%
    \@ifnextchar\nolimits{%
      \mint@l{#1}%
    }{%
      \@ifnextchar\displaylimits{%
        \mint@l{#1}%
      }{%
        \mint@s{#2}{#1}%
      }%
    }%
  }%
}
\newcommand*{\mint@s}[2]{%
  % #1: limits
  % #2: overlay symbol
  \@ifnextchar_{%
    \mint@sub{#1}{#2}%
  }{%
    \@ifnextchar^{%
      \mint@sup{#1}{#2}%
    }{%
      \mint@{#1}{#2}{}{}%
    }%
  }%
}
\def\mint@sub#1#2_#3{%
  \@ifnextchar^{%
    \mint@sub@sup{#1}{#2}{#3}%
  }{%
    \mint@{#1}{#2}{#3}{}%
  }%
}
\def\mint@sup#1#2^#3{%
  \@ifnextchar_{%
    \mint@sup@sub{#1}{#2}{#3}%
  }{%
    \mint@{#1}{#2}{}{#3}%
  }%
}
\def\mint@sub@sup#1#2#3^#4{%
  \mint@{#1}{#2}{#3}{#4}%
}
\def\mint@sup@sub#1#2#3_#4{%
  \mint@{#1}{#2}{#4}{#3}%
}
\newcommand*{\mint@}[4]{%
  % #1: \limits, \nolimits, \displaylimits
  % #2: overlay symbol: -, =, ...
  % #3: subscript
  % #4: superscript
  \mathop{}%
  \mkern-\thinmuskip
  \mathchoice{%
    \mint@@{#1}{#2}{#3}{#4}%
        \displaystyle\textstyle\scriptstyle
  }{%
    \mint@@{#1}{#2}{#3}{#4}%
        \textstyle\scriptstyle\scriptstyle
  }{%
    \mint@@{#1}{#2}{#3}{#4}%
        \scriptstyle\scriptscriptstyle\scriptscriptstyle
  }{%
    \mint@@{#1}{#2}{#3}{#4}%
        \scriptscriptstyle\scriptscriptstyle\scriptscriptstyle
  }%
  \mkern-\thinmuskip
  \int#1%
  \ifx\\#3\\\else_{#3}\fi
  \ifx\\#4\\\else^{#4}\fi  
}
\newcommand*{\mint@@}[7]{%
  % #1: limits
  % #2: overlay symbol
  % #3: subscript
  % #4: superscript
  % #5: math style
  % #6: math style for overlay symbol
  % #7: math style for subscript/superscript
  \begingroup
    \sbox0{$#5\int\m@th$}%
    \sbox2{$#5\int_{}\m@th$}%
    \dimen2=\wd0 %
    % => \dimen2 = width of \int
    \let\mint@limits=#1\relax
    \ifx\mint@limits\relax
      \sbox4{$#5\int_{\kern1sp}^{\kern1sp}\m@th$}%
      \ifdim\wd4>\wd2 %
        \let\mint@limits=\nolimits
      \else
        \let\mint@limits=\limits
      \fi
    \fi
    \ifx\mint@limits\displaylimits
      \ifx#5\displaystyle
        \let\mint@limits=\limits
      \fi
    \fi
    \ifx\mint@limits\limits
      \sbox0{$#7#3\m@th$}%
      \sbox2{$#7#4\m@th$}%
      \ifdim\wd0>\dimen2 %
        \dimen2=\wd0 %
      \fi
      \ifdim\wd2>\dimen2 %
        \dimen2=\wd2 %
      \fi
    \fi
    \rlap{%
      $#5%
        \vcenter{%
          \hbox to\dimen2{%
            \hss
            $#6{#2}\m@th$%
            \hss
          }%
        }%
      $%
    }%
  \endgroup
}

%----------------------------------------------------------------------------------------
%	APPENDICE
%----------------------------------------------------------------------------------------

\usepackage[toc,page]{appendix}

%----------------------------------------------------------------------------------------
%	SIMBOLI CARTE DA GIOCO
%
% Sono i seguenti:
%   \varheartsuit
%   \vardiamondsuit
%   \clubsuit
%   \spadesuit
%----------------------------------------------------------------------------------------

\DeclareSymbolFont{extraup}{U}{zavm}{m}{n}
\DeclareMathSymbol{\varheartsuit}{\mathalpha}{extraup}{86}
\DeclareMathSymbol{\vardiamondsuit}{\mathalpha}{extraup}{87}


%----------------------------------------------------------------------------------------
%----------------------------------------------------------------------------------------
%----------------------------------------------------------------------------------------
%----------------------------------------------------------------------------------------
%----------------------------------------------------------------------------------------
%----------------------------------------------------------------------------------------
%----------------------------------------------------------------------------------------
%----------------------------------------------------------------------------------------
%----------------------------------------------------------------------------------------
%----------------------------------------------------------------------------------------
%%%% TESTING

\definecolor{grey245}{RGB}{245,245,245}

% questo fa si che le footnote siano "fuori" dall'environment dimostrazione
% \usepackage{footnote}
% \usepackage{etoolbox}
% \BeforeBeginEnvironment{dimostrazione}{\savenotes}
% \AfterEndEnvironment{dimostrazione}{\spewnotes}
% \BeforeBeginEnvironment{theorem}{\savenotes}
% \AfterEndEnvironment{theorem}{\spewnotes}

\newtheoremstyle{blacknumbox} % Theorem style name
{0pt}% Space above
{0pt}% Space below
{\normalfont}% Body font
{}% Indent amount
{\bf\scshape}% Theorem head font --- {\small\bf}
{.\;}% Punctuation after theorem head
{0.25em}% Space after theorem head
{\small\thmname{#1}\nobreakspace\thmnumber{\@ifnotempty{#1}{}\@upn{#2}}% Theorem text (e.g. Theorem 2.1)
%{\small\thmname{#1}% Theorem text (e.g. Theorem)
\thmnote{\nobreakspace\the\thm@notefont\normalfont\bfseries---\nobreakspace#3}}% Optional theorem note

\newtheoremstyle{unnumbered} % Theorem style name
{0pt}% Space above
{0pt}% Space below
{\normalfont}% Body font
{}% Indent amount
{\bf\scshape}% Theorem head font --- {\small\bf}
{.\;}% Punctuation after theorem head
{0.25em}% Space after theorem head
{\small\thmname{#1}\thmnumber{\@ifnotempty{#1}{}\@upn{#2}}% Theorem text (e.g. Theorem 2.1)
%{\small\thmname{#1}% Theorem text (e.g. Theorem)
\thmnote{\nobreakspace\the\thm@notefont\normalfont\bfseries---\nobreakspace#3}}% Optional theorem note


\newcounter{dummy} 
\numberwithin{dummy}{chapter}


\theoremstyle{blacknumbox}
\newtheorem{definitionT}[dummy]{Definition}
\newtheorem{theoremT}[dummy]{Theorem}
\newtheorem{corollarioT}[dummy]{Corollary}
\newtheorem{lemmaT}[dummy]{Lemma}

% Per gli unnumbered tolgo il \nobreakspace subito dopo {\small\thmname{#1} perché altrimenti c'è uno spazio tra Teorema e il ".", lo spazio lo voglio solo se sono numerati per distanziare Teorema e "(2.1)"
\theoremstyle{unnumbered}
\newtheorem*{NBT}{NB}
\newtheorem*{ossT}{Remark}
\newtheorem*{ricalgT}{Richiamo di Algebra}
\newtheorem*{pseudocodiceT}{Pseudocodice}
\newtheorem*{dimT}{Proof}

\RequirePackage[framemethod=default]{mdframed} % Required for creating the theorem, definition, exercise and corollary boxes

\usepackage{lipsum}



% orange box
\newmdenv[skipabove=7pt,
skipbelow=7pt,
rightline=false,
leftline=true,
topline=false,
bottomline=false,
linecolor=orange,
backgroundcolor=orange!5,
innerleftmargin=5pt,
innerrightmargin=5pt,
innertopmargin=5pt,
leftmargin=0cm,
rightmargin=0cm,
linewidth=2pt,
innerbottommargin=5pt]{oBox}

% green box
\newmdenv[skipabove=7pt,
skipbelow=7pt,
rightline=false,
leftline=true,
topline=false,
bottomline=false,
linecolor=green,
backgroundcolor=green!5,
innerleftmargin=5pt,
innerrightmargin=5pt,
innertopmargin=5pt,
leftmargin=0cm,
rightmargin=0cm,
linewidth=2pt,
innerbottommargin=5pt]{gBox}

% blue box
\newmdenv[skipabove=7pt,
skipbelow=7pt,
rightline=false,
leftline=true,
topline=false,
bottomline=false,
linecolor=blue,
backgroundcolor=blue!5,
innerleftmargin=5pt,
innerrightmargin=5pt,
innertopmargin=5pt,
leftmargin=0cm,
rightmargin=0cm,
linewidth=2pt,
innerbottommargin=5pt]{bBox}

% dim box
\newmdenv[skipabove=7pt,
skipbelow=7pt,
rightline=false,
leftline=true,
topline=false,
bottomline=false,
linecolor=black,
backgroundcolor=grey245!0,
innerleftmargin=5pt,
innerrightmargin=5pt,
innertopmargin=5pt,
leftmargin=0cm,
rightmargin=0cm,
linewidth=2pt,
innerbottommargin=5pt]{dimBox}

\newenvironment{theorem}{\begin{gBox}\begin{theoremT}}{\end{theoremT}\end{gBox}}	
\newenvironment{definition}{\begin{bBox}\begin{definitionT}}{\end{definitionT}\end{bBox}}	
\newenvironment{nb}{\begin{oBox}\begin{NBT}}{\end{NBT}\end{oBox}}	
\newenvironment{rmk}{\begin{oBox}\begin{ossT}}{\end{ossT}\end{oBox}}
\newenvironment{ricalg}{\begin{oBox}\begin{ricalgT}}{\end{ricalgT}\end{oBox}}
\newenvironment{pseudocode}{\begin{oBox}\begin{pseudocodiceT}}{\end{pseudocodiceT}\end{oBox}}
\newenvironment{corollary}{\begin{oBox}\begin{corollarioT}}{\end{corollarioT}\end{oBox}}
\newenvironment{lemma}{\begin{oBox}\begin{lemmaT}}{\end{lemmaT}\end{oBox}}
\newenvironment{prf}{\begin{dimBox}\begin{dimT}}{\[\qed\]\end{dimT}\end{dimBox}}

%----------------------------------------------------------------------------------------
%	INDICE INIZIALE
%----------------------------------------------------------------------------------------

\setcounter{secnumdepth}{3} % DI DEFAULT LE SUBSUBSECTION NON SONO NUMERATE, COSÌ SÌ
\setcounter{tocdepth}{2} % FISSA LA PROFONDITÀ DELLE COSE MOSTRATE NELL'INDICE

\usepackage[hidelinks]{hyperref} % Rende l'indice interattivo e hidelinks nasconde il bordo rosso dai riferimenti

%% Questo fa si che la numerazione dei capitoli venga resettata se si inizia una nuova "Parte"
%\makeatletter
%\@addtoreset{chapter}{part}
%\makeatother 

\newcommand{\nocontentsline}[3]{} % QUESTO COMANDO E QUELLO DOPO SERVONO PER AVERE IL COMANDO \tocless DA METTERE PRIMA DI UNA SEZIONE CHE NON VOGLIO FAR APPARIRE NELL'INDICE
\newcommand{\tocless}[2]{\bgroup\let\addcontentsline=\nocontentsline#1{#2}\egroup}




\usepackage{geometry}
\geometry{
	nomarginpar, % Toglie doppi margini
	margin=1in, % Imposta i margini a 1 inch
}

% Comandi pratici

% Lettere comuni in grassetto
\newcommand{\x}{\mathbf{x}}
\newcommand{\y}{\mathbf{y}}
\newcommand{\zer}{\mathbf{0}}

% Lettere comuni in grassetto greche
\usepackage{bm} % per avere le lettere greche in bold con \bm{\sigma}
\newcommand{\sigg}{\bm{\sigma}}
\newcommand{\nuu}{\bm{\nu}}
\newcommand{\alphaa}{\bm{\alpha}}

% d nell'integrale e i rispettivi usi
\newcommand{\de}{\,\mathrm d}
\newcommand{\dx}{\de x}
\newcommand{\dy}{\de y}
\newcommand{\dl}{\de l}
\newcommand{\dr}{\de r}
\newcommand{\ds}{\de s}
\newcommand{\dt}{\de t}
\newcommand{\dv}{\de v}
\newcommand{\dxi}{\de \xi}
\newcommand{\drho}{\de \rho}

% d nell'integrale con differenziale vettoriale
\newcommand{\dxx}{\de \x}
\newcommand{\dyy}{\de \y}
\newcommand{\dsig}{\de \sigg}

\newcounter{es}
\newcommand{\Es}{
	\stepcounter{es}
	\section{Esercizio \arabic{es}}
	}

\newcommand{\Par}{\textbf{Parametri}}
\newcommand{\Var}{\textbf{Variabili}}
\newcommand{\Fob}{\textbf{Funzione obiettivo}}
\newcommand{\Vin}{\textbf{Vincoli}}

% Non servono, preferisco la numerazione locale e non globale delle figure e delle equazioni
%\usepackage{chngcntr}
%\counterwithout{figure}{chapter}
%\counterwithout{equation}{chapter}

%\allowdisplaybreaks[4] % Consente di rompere equazioni su più pagine

%%%%%%%%%%%%%%%%%%%%%%%%%%%%%%%%%%%%%%%%%%%%%%%
%%%%%%%%%%%%%%%%%%%%%%%%%%%%%%%%%%%%%%%%%%%%%%%

\begin{document}

%%%%%%%%%%%%%%%%%%%%%%%%%%%%%%%%%%%%%%%%%%%%%%%
%%%%%%%%%%%%%%%%%%%%%%%%%%%%%%%%%%%%%%%%%%%%%%%

\frontmatter
\pagestyle{empty} % SWITCHA PER NON AVERE NUMERO PAGINA
\vspace*{\fill}
\begin{center}
	{\large \textsc{Notes of}}\\
	\vspace*{0.4cm}
	{\Huge \textsc{Game Theory}}\\
	\vspace*{1cm}
	{\large {From Prof. Gianni Arioli's lectures}}\\
	\vspace*{0.1cm}
	{\large for the MSc in Mathematical Engineering}\\
	\vspace*{0.4cm}
	{\large {by Teo Bucci}}\\
	\vspace*{1cm}
	Politecnico di Milano\\A.Y. 2021/2022
\end{center}
\vspace*{\fill}
\newpage

%%%%%%%%%%%%%%%%%%%%%%%%%%%%%%%%%%%%%%%%%%%%%%%
%%%%%%%%%%%%%%%%%%%%%%%%%%%%%%%%%%%%%%%%%%%%%%%

%{\Large \textit{Appunti di Equazioni alle Derivate Parziali}}
%
%\vspace*{\fill}
%
%\textcopyright \ Gli autori, tutti i diritti riservati
%
%Sono proibite tutte le riproduzioni senza autorizzazione scritta degli autori.
%
%Revisione del \today
%
%Developed by\\
%Teo Bucci - \texttt{teobucci8@gmail.com}\\
%Filippo Cipriani - \texttt{filippo.cipriani99@hotmail.it}\\ \\
%Compiled with \ensuremath\heartsuit \\

%\textbf{Prefazione}

%Per segnalare eventuali errori o suggerimenti potete contattare gli autori.

\newpage

%%%%%%%%%%%%%%%%%%%%%%%%%%%%%%%%%%%%%%%%%%%%%%%
%%%%%%%%%%%%%%%%%%%%%%%%%%%%%%%%%%%%%%%%%%%%%%%

% INDICE
\addtocontents{toc}{\protect\thispagestyle{empty}}
\tableofcontents
%\newpage

%%%%%%%%%%%%%%%%%%%%%%%%%%%%%%%%%%%%%%%%%%%%%%%
%%%%%%%%%%%%%%%%%%%%%%%%%%%%%%%%%%%%%%%%%%%%%%%

% PAGINA VUOTA PER FAR PARTIRE IL CAPITOLO IN UNA PAGINA DISPARI
%\myNewEmptyPage

\AtEndDocument{\cleardoublepage}

%%%%%%%%%%%%%%%%%%%%%%%%%%%%%%%%%%%%%%%%%%%%%%%
%%%%%%%%%%%%%%%%%%%%%%%%%%%%%%%%%%%%%%%%%%%%%%%

\mainmatter
\pagestyle{fancy} % Riswitcha per riavere il numero pagina
%\setcounter{page}{1} % Fa ripartire il contatore pagina da 1

%%%%%%%%%%%%%%%%%%%%%%%%%%%%%%%%%%%%%%%%%%%%%%%
%%%%%%%%%%%%%%%%%%%%%%%%%%%%%%%%%%%%%%%%%%%%%%%

\chapter{Programmazione Lineare}

\Es

\Par

$P$ porti, $i=1,2,3$

$c_{i}$ costo per porto per ogni vettura $( 150,250,200)$

$t_{i}$ costo fisso porto

$S$ centri di smistamento, $j=1,\dotsc ,4$

$k_{i}$ costo di invio dal porto $i$ al km

$a_{ij}$ distanza dal porto $i$ al centro $j$

$r_{j}$ richiesta del centro $j$

$d_{i}$ capacità del porto $i$

\Var

$x_{ij} \geq 0,x_{ij} \in \mathbb{Z}$ numero di automobili dal porto $i$ al centro $j$

$y_{i} \in \{0,1\}$, uguali a $1$ se uso il porto $i$

$z_{ij} \in \{0,1\}$, uguali a $1$ se il porto $i$ rifornisce il centro $j$

\Fob

\begin{equation*}
	\min\bigg\{\underbrace{\sum _{ij} c_{i} x_{ij}}_{\text{auto}} +\underbrace{\sum _{i} t_{i} y_{i}}_{\text{porto}} +\underbrace{\sum _{ij} a_{ij} k_{i} x_{ij}}_{\text{trasporto}}\bigg\}
\end{equation*}

\Vin

$\sum _{i} x_{ij} \geq r_{j} ,\forall j\in S$

$\sum _{j} x_{ij} \leq d_{i} y_{i} ,\forall i\in P$

$\sum _{i} z_{i,3} =1$

$x_{ij} \leq d_{i} z_{ij} ,\forall i\in P,\forall j\in S$

$z_{22} \leq z_{24}$

\Es

\Par

$A$ aeroporti

$H$ hangar

$c_{j},s_{j},t_{j}$ operatori $\forall j\in H$

$g_{1}$ costo squadra $1$

$g_{2}$ costo squadra $2$

$g_{3}$ costo squadra $3$

\begin{center}
	
	\begin{tabular}{|c|c|c|}
		\hline 
		$1$c & $1$s & $1$t \\
		\hline 
		$3$c & $1$s & X    \\
		\hline 
		$3$c & $2$s & $2$t \\
		\hline
	\end{tabular}
\end{center}

\Var

$x_{j} \geqslant 0,x_{j} \in \mathbb{Z}$ squadre tipo $1$

$y_{j} \geqslant 0,y_{j} \in \mathbb{Z}$ squadre tipo $2$

$z_{j} \geqslant 0,z_{j} \in \mathbb{Z}$ squadre tipo $3$

$\varphi \in \{0,1\}$, uguale a $1$ se uso 3 squadre di tipo $2$

$w_{ij} \in \{0,1\}$, uguale a $1$ se aereo $i$ in hangar $j$, $\forall i\in A,\forall j\in H$

\Fob

\begin{equation*}
	\min \sum _{j}( x_{j} g_{1} +y_{j} g_{2} +z_{j} g_{3})
\end{equation*}

\Vin

$\sum _{j} w_{ij} =1,\forall i\in A$

$\begin{drcases}
	\begin{array}{ c c c c c }
		x_{j} & +3y_{j} & +3z_{j} & \geq \sum _{i} c_{j} w_{ij} & \forall j\in H \\
		x_{j} & +y_{j}  & +2z_{j} & \geq \sum _{i} s_{j} w_{ij} & \forall j\in H \\
		x_{j} &         & +z_{j}  & \geq \sum _{i} t_{j} w_{ij} & \forall j\in H 
	\end{array}
\end{drcases} \ \text{operai}$

$y_{j} \geqslant 3\ \ \overset{\text{(A)}}{\Rightarrow } \ \ \varphi =1\ \ \overset{\text{(B)}}{\Rightarrow } \ \ z_{j} \geqslant 2$

(A) $\sum _{j} y_{j} -2\leq M\varphi $

(B) $2\varphi \leq \sum _{j} z_{j}$

\Es

\Par

$p_{j} ,\ j=1,2$

$r_{j}$ pretto vendita

$d_{j}$ domanda

$I$ materie prime $i\in I$

$c_{i}$ disponibilità

$g_{i}$ costo unitario materie prima

$g_{ji}$ materia $i$ necessaria per $j$

$o_{1}$ ore $p_{1}$ da materia prima

$o_{2}$ ore $p_{2}$ da materia prima

oppure ottengo $p_{2}$ con

$b$ unità di $p_{1}$ per $p_{2}$

$o_{3}$ ore lavorazione ($p_{2}$ da $p_{1}$)

$k$ costo fisso attivazione

$O$ ore a disposizione

\Var

$x_{j} \geqslant 0,x_{j} \in \mathbb{Z}$ unità di prodotto $j$ da materie prime

$y\geq 0,y\in \mathbb{Z}$ unità di prodotto $2$ da prodotto $1$

$z\in \{0,1\}$, uguale a $1$ se attivo processo produttivo\Fob

\begin{equation*}
	\max\left\{[ r_{1}( x_{1} -by) +r_{2}( x_{2} +y)] -\left[\sum _{ij} g_{i} q_{ji} x_{j} +kz\right]\right\}
\end{equation*}

\Vin

$y\leq Mz$

$( x_{1} -by) \geqslant d_{1}$

$( x_{2} -y) \geqslant d_{2}$

$\sum _{j} q_{ji} x_{j} \leq c_{i} ,\forall i\in I$

$o_{1} x_{1} +o_{2} x_{2} +o_{3} y\leq O$

\Es

\Par

$T$ gruppi $i\in T$

$p_{i}$ persone

$J$ aerei $j\in J$

$c_{j}$ costo noleggio

$B_{j}$ capienza aereo

$A$ aeroporto $k\in A$

$G_{k}$ max voli per aeroporto

$l_{jk}$ costo di far partire $j$ da $k$

$R$ sottoinsiemi di aeroporti vicini

$S_{r}$ con $r=1,\dotsc ,R$, al più un aeroporto

\Var

$x_{ij} \in \{0,1\}$, uguale a $1$ se gruppo $i$ ad aereo $j$

$y_{jk} \in \{0,1\}$, uguale a $1$ se aereo $j$ parte da $k$

$z_{j} \in \{0,1\}$, uguale a $1$ se uso aereo $j$

$w_{k} \in \{0,1\}$, uguale a $1$ se uso aeroporto $k$

\Fob

\begin{equation*}
	\min\left\{\sum _{j} c_{j} z_{j} +\sum _{jk} l_{jk} y_{jk}\right\}
\end{equation*}

\Vin

$\sum _{i} x_{ij} \leq Mz_{j} ,\forall j\in J$

$\sum _{i} p_{i} x_{ij} \leq B_{j} ,\forall j\in J$

$\sum _{j} y_{jk} \leq G_{k} w_{k} ,\forall k\in K$

$\sum _{k\in S_{r}} w_{k} \leq 1,\forall r=1,\dotsc ,R$

$\sum _{j} x_{ij} =1,\forall i\in I$

$\sum _{k} y_{jk} =z_{j} ,\forall j\in J$

\Es

\Par

$P$ domande iscrizione $i\in P$

$M\subset P,F\subset P$, uomini, donne $( M\cup F=P,M\cap F=\emptyset )$

$n$ max persone per classe

$d$ massimo classi ($D=1,\dotsc ,d$ insieme classi)

$b_{i}$ preparazione di $i$

$q$ livello minimo per classe

$C$ coppie formate $( i,j) \in C,i\in M,j\in F$

\Var

$x_{ik} \in \{0,1\}$, uguale a $1$ se persona $i$ in classe $k$

$y_{i} \in \{0,1\}$, uguale a $1$ se accetto domanda

\Fob

\begin{equation*}
	\max\sum _{i} y_{i}
\end{equation*}

\Vin

$\sum _{i\in P} x_{ik} \leq n,\forall k\in D$ capacità classe

$\sum _{i\in M} x_{ik} =\sum _{i\in F} x_{ik} ,\forall k\in D$ uguali $M/F$

$\sum _{i\in P} x_{ik} b_{i} \geqslant q\sum _{i\in P} x_{ik} ,\forall k\in D$ preparazione

$y_{i} \leq \sum _{k\in D} x_{ik} ,\forall i\in P$ bigM

$\sum _{k\in D} x_{ik} \leq 1,\forall i\in P$ massimo $1$ corso per persona

$x_{ik} =x_{jk} ,\forall (i,j)\in C,\forall k\in D$ coppie

\Es

\Par

$A$ insieme altiforni $i=1\dotsc N,i\in A$

$m_{i}$ max quintali per altiforno

$P$ prodotti $j\in P$

$q_{1j}$ prodotto $j$ da 1 quintale di materia prime con processo 1 (prodotto/quintale)

$q_{2j}$ prodotto $j$ da 1 quintale di materia prime con processo 2 (prodotto/quintale)

$r_{j}$ richiesto prodotto

$c_{1i}$ costo lavorazione al quintale in altiforno $i$ con processo 1 (euro/quintale)

$c_{2i}$ costo lavorazione al quintale in altiforno $i$ con processo $2$ (euro/quintale)

$f_{i}$ costo attivazione processo 2 in altiforno $i$\Var

$w_{i} \in \{0,1\}$, uguale a $1$ se lavoro più di $q$

$y_{i} \in \{0,1\}$, uguale a $1$ se uso processo $2$

$x_{ij1} \geqslant 0,x_{ij1} \in \mathbb{Z}$ prodotto $j$ con processo $1$ in altiforno $i$

$x_{ij2} \geqslant 0,x_{ij2} \in \mathbb{Z}$ prodotto $j$ con processo $2$ in altiforno $i$

\Fob

\begin{equation*}
	\min \left\{\sum _{i} y_{i} f_{i} +\sum _{ij}\left[ c_{1i}\frac{x_{ij1}}{q_{1j}} +c_{2i}\frac{x_{ij2}}{q_{2j}}\right]\right\}
\end{equation*}

\Vin

$\sum _{j} x_{ij2} \leq My_{i} ,\forall i\in A$ bigM

$\sum _{j}\left[\frac{x_{ij1}}{q_{1j}} +\frac{x_{ij2}}{q_{2j}}\right] \leq m_{i} ,\forall i\in A$ capacità

$\sum _{i}[ x_{ij1} +x_{ij2}] \geqslant r_{j} ,\forall j\in P$ richiesta

$\sum _{i} y_{i} \leq N-1$ no processo $2$ su tutti gli altiforni

$\sum _{i} w_{i} \geqslant 1$ almeno 1 usa più di $q$ quintali

$qw_{i} \leq \sum _{ij}\left[\frac{x_{ij1}}{q_{1j}} +\frac{x_{ij2}}{q_{2j}}\right] ,\forall i\in A$ vincolo logico

\Es

\Par

$C$ cioccolatini $i\in C$

$S$ confezioni regalo $j\in S$

$r_{ij}$ richieste cioccolatini $i$ in confezione $j$

$g_{i}$ costo cioccolatino

$m_{i}$ max produzione

$p_{i}$ vendita cioccolatino sfuso $i$

$d_{j}$ vendita confezione $j$

$b_{j}$ costo scatola $j$\Var

$x_{i} \geqslant 0,x_{i} \in \mathbb{Z}$ numero cioccolatini $i$ prodotti

$y_{j} \geqslant 0,y_{j} \in \mathbb{Z}$ numero confezioni $j$ prodotte

$z\in \{0,1\}$, uguale a $1$ se acquisto almeno $q$ scatole

\Fob

\begin{equation*}
	\max\bigg\{\underbrace{\sum _{j} d_{j} y_{j}}_{\text{confezioni}} +\underbrace{\sum _{i} p_{i}\bigg( x_{i} -\sum _{j} r_{ij} y_{j}\bigg)}_{\text{sfusi}} -\underbrace{\sum _{i} g_{i} x_{i}}_{\text{costo prod.}} -\underbrace{\sum _{j} b_{j} y_{j}}_{\text{costo scatole}} +\underbrace{zB}_{\text{sconto}}\bigg\}
\end{equation*}

\Vin

$x_{i} \geqslant \sum _{j} r_{ij} y_{j} ,\forall i\in I$ richiesta

$x_{i} \leq m_{i} ,\forall i\in I$ capacità

$\sum _{j} y_{j} \geqslant Qz$ sconto

$x_{1} \geqslant 0.2\cdot \sum _{i} x_{i} \ \ $ quolità

\Es

\Par

$D$ difensori

$A$ attaccanti

$G$ giocatori $i\in G$

$r_{i} \in \{0,1\}$, uguale a $1$ se giocatore $i$ è attaccante

$v_{i}$ valore giocatore

$B$ valore complessivo formazione

$q$ giocatori non giocanti

$K$ formazioni $|K|=2$

\Var

$z\geqslant 0,z\in \mathbb{Z}$ valore formazione di minimo valore

$x_{ik} \in \{0,1\}$, uguale a $1$ se giocatore $i$ è nelle formazione $k$

$y_{i} \in \{0,1\}$, uguale a $1$ se $i$ gioca in entrambe

\Fob

\begin{equation*}
	\max z
\end{equation*}

\Vin

$\sum _{i} r_{i} x_{ik} =A,\forall k\in K$

$\sum _{i}( 1-r_{i}) x_{ik} =D,\forall k\in K$

$\sum _{i} v_{i} x_{ik} \geqslant B,\forall k\in K$ minimo valore richiesto

$\left( |G|-\sum _{i} y_{i}\right) \geqslant q$ almeno $q$ non giocanti entrambe

$\left(\sum _{k} x_{ik} -1\right) \leq My_{i} ,\forall i\in I$ bigM

$z\leq \sum _{i} v_{i} x_{ik} ,\forall k\in K$ bottleneck

\Es

\Par

$B$ beni $i\in B$

$M$ magazzino $j\in M$

$A$ luoghi distribuzione $k\in A$

$c_{i}$ costo bene $i$

$v_{i}$ spazio occupato da $i$ in magazzino

$b_{j}$ capacità

$f_{j}$ costo fisso magazzino se usato

$g_{jk}$ costo trasporto bene da $j$ a $k$

$d_{ik}$ richiesta bene $i$ a $k$

\Var

$y_{j} \in \{0,1\}$, uguale a $1$ se uso $j$

$z_{ijk} \geqslant 0,z_{ijk} \in \mathbb{Z}$ numero di beni $i$ da $j$ a $k$

\Fob

\begin{equation*}
	\min\left\{\sum _{ijk} c_{i} z_{ijk} +\sum _{j} f_{j} y_{j} +\sum _{ijk} z_{ijk} g_{jk}\right\}
\end{equation*}

\Vin

$\sum _{j} z_{ijk} \geqslant d_{ik} ,\forall i\in I,\forall k\in K$ richiesta

$\sum _{ik} v_{i} z_{ijk} \leq b_{j} y_{j} ,\forall j\in J$ bigM e capacità

\Es

\Par

$C$ analisi $i\in C,i=1,\dotsc ,4$

$O$ ospedali $j\in O,j=1,\dotsc ,5$

$d_{ij}$ tempo da $i$ a $j$

$r_{j}$ richieste analisi

$b_{i}$ max analisi nel centro $i$

\Var

$x_{ij} \geqslant 0,x_{ij} \in \mathbb{Z}$ numero analisi al centro $i$ per ospedale $j$

$z_{2i} \in \{0,1\}$, uguale a $1$ se 2 si serve da $i$

\Fob

\begin{equation*}
	\min\sum _{ij} a_{ij} x_{ij}
\end{equation*}

\Vin

${\textstyle \begin{drcases}
	{\textstyle \sum\nolimits _{j} x_{1j} \leq 0.8\cdot \left(\sum\nolimits _{j} x_{2j} +x_{3j}\right)}\\
	{\textstyle \sum _{j} x_{2j} \leq 0.6\cdot \left(\sum _{j} x_{ij} +x_{3j}\right)}\\
	{\textstyle \sum _{j}( x_{3j} +x_{4j}) \leq 0.5\cdot \sum _{ij} x_{ij}}
	\end{drcases} \ \text{qualità}}$

$\sum _{i} x_{ij} =r_{j} ,\forall j\in J$ richiesta

$\sum _{j} x_{ij} \leq b_{i} ,\forall i\in I$ capacità

$\sum z_{2i} =1$

$x_{i2} \leq b_{i} z_{2i} ,\forall i\in I$ bigM
\end{document}